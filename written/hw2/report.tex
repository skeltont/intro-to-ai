\documentclass[10pt,draftclsnofoot,onecolumn]{IEEEtran}

\usepackage{setspace}
\usepackage{listings}
\usepackage{cite}
\usepackage{enumerate}
\usepackage{amsthm}
\usepackage[fleqn]{amsmath}
% \usepackage{chngcntr}
% \usepackage{mathtools}

% \numberwithin{equation}

% set syntax highlighting for listings
\lstset{language=C}
\lstset{frame=lrbt,xleftmargin=\fboxsep,xrightmargin=-\fboxsep}

% correct bad hyphenation here
\hyphenation{op-tical net-works semi-conduc-tor}

\begin{document}

\pagenumbering{gobble} % hide page number on first page
\singlespacing % set spacing
\title{Written Assignment \#2}

\author{Ty~Skelton}

% make the title area
\maketitle

\begin{center}
\scshape                      % Small caps
CS331 - Intro to AI \\        % Course
Spring Term\\[\baselineskip]  % Term
Oregon State University\par   % Location
\end{center}

\IEEEpeerreviewmaketitle

\pagenumbering{arabic}
\newpage

\begin{enumerate}
  \item (From 7.4 in the book) For each of the following statements, prove if it is true or false.
  \begin{enumerate}[a)]
    \item \( (A \land B) \models (A \iff B) \) \\

    \begin{tabular}{|c|c|c|c|}
      \hline
      \( A \) & \( B \) & \( A \land B \) & \( A \iff B \) \\
      \hline
      T & T & T & T \\
      T & F & F & F \\
      F & T & F & F \\
      F & F & F & T \\
      \hline
    \end{tabular} \medskip % readability

    Since \( (A \iff B) \) is true whenever \( (A \land B) \) is true, then \( (A \land B) \models (A \iff B) \) is therefore true. \\

    \item \( (C \lor (\neg A \land \neg B)) \equiv ((A \rightarrow C) \land (B \rightarrow C)) \) becomes \( (C \lor (\neg A \land \neg B)) \equiv ((\neg A \lor C) \land (\neg B \lor C)) \) \\

    \begin{tabular}{|c|c|c|c|c|}
      \hline
      \( A \) & \( B \) & \( C \) & \( C \lor (\neg A \land \neg B) \) & \( (\neg A \lor C) \land (\neg B \lor C) \) \\
      \hline
      T & T & T & T & T \\
      T & T & F & F & F \\
      T & F & F & F & F \\
      T & F & T & T & T \\
      F & T & T & T & T \\
      F & T & F & F & F \\
      F & F & T & T & T \\
      F & F & F & T & T \\
      \hline
    \end{tabular} \medskip % readability

    Since \( (C \lor (\neg A \land \neg B)) \) has the same truth values as \( ((\neg A \lor C) \land (\neg B \lor C)) \), then \( (C \lor (\neg A \land \neg B)) \equiv ((\neg A \lor C) \land (\neg B \lor C)) \) is therefore true. \\

    \item \( (A \lor B) \land \neg (A \rightarrow B) \) becomes \( (A \lor B) \land (A \land \neg B) \) \\

    \begin{tabular}{|c|c|c|}
      \hline
      \( A \) & \( B \) & \(  (A \lor B) \land (A \land \neg B) \)  \\
      \hline
      T & T & F \\
      T & F & T \\
      F & T & F \\
      F & F & F \\
      \hline
    \end{tabular} \medskip % readability

    Since when \( A \) is true and \( B \) is false \(  (A \lor B) \land (A \land \neg B) \) is evaluated to true, then it is satisfiable.

  \end{enumerate}

  \item (From 7.10 in the book) Decide whether each of the following sentences is valid, unsatisfiable or neither.
  Verify your decisions using truth tables or the equivalence rules of Figure 7.11.
  \begin{enumerate}[a)]
    \item \( Smoke \rightarrow Smoke \) becomes \( \neg Smoke \lor Smoke \) \\

    \begin{tabular}{|c|c|}
      \hline
      \( Smoke \) & \( \neg Smoke \lor Smoke \)  \\
      \hline
      T & T \\
      F & T \\
      \hline
    \end{tabular} \medskip % readability

    This statement is satisfiable and valid as shown in the truth table for \( Smoke \) above. \\

    \item \( (Smoke \rightarrow Fire) \rightarrow (\neg Smoke \rightarrow \neg Fire) \) becomes \( (Smoke \land \neg Fire) \lor (Smoke \lor \neg Fire) \) \\

    \begin{tabular}{|c|c|c|}
      \hline
      \( Smoke \) & \( Fire \) & \( (Smoke \land \neg Fire) \lor (Smoke \lor \neg Fire) \)  \\
      \hline
      T & T & T \\
      T & F & T \\
      F & T & F \\
      F & F & T \\
      \hline
    \end{tabular} \medskip % readability

    This statement is satisfiable and valid as shown in the truth table for \( Smoke \) and \( Fire \) above. \\

    \newpage % readability

    \item \( (Smoke \rightarrow Fire) \rightarrow ((Smoke \land Heat) \rightarrow Fire) \) becomes \( (Smoke \land \neg Fire) \lor ((\neg Smoke \lor \neg Heat) \lor Fire) \) \\

    \begin{tabular}{|c|c|c|c|}
      \hline
      \( Smoke \) & \( Fire \) & \( Heat \) & \( (Smoke \land \neg Fire) \lor ((\neg Smoke \lor \neg Heat) \lor Fire) \) \\
      \hline
      T & T & T & T \\
      T & T & F & T \\
      T & F & F & T \\
      T & F & T & T \\
      F & T & T & T \\
      F & T & F & T \\
      F & F & T & T \\
      F & F & F & T \\
      \hline
    \end{tabular} \medskip % readability

    Since every possible value for \( Smoke \), \( Fire \), and \( Heat \) results in a true statement for \( (Smoke \land \neg Fire) \lor ((\neg Smoke \lor \neg Heat) \lor Fire) \) it is valid. \\

  \end{enumerate}


  \item  (Exercise 7.2 in the book which was adapted from Barwise and Etchemendy (1993)).
  If a unicorn is mythical, then it is immortal, but if it is not mythical, then it is a mortal mammal.
  If the unicorn is either immortal or a mammal, then it is horned.
  The unicorn is magical if it is horned.
  \begin{enumerate}[a)]
    \item Knowledge Base:
    \begin{align}
      Mythical &\rightarrow Immortal \\
      \neg Mythical &\rightarrow (\neg Immortal \land Mammal) \\
      (Immortal \lor Mammal) &\rightarrow Horned \\
      Horned &\rightarrow Magical
    \end{align}

    \begin{align}
      \neg Mythical &\lor Immortal &[1, logical \; equiv.] \\
      Mythical &\lor (\neg Immortal \land Mammal) &[2, logical \; equiv.] \\
      (\neg Immortal \land \neg Mammal) &\lor Horned &[3, logical \; equiv.] \\
      \neg Horned &\lor Magical &[4, logical \; equiv.] \\
      (\neg Immortal \land \neg Mammal) &\lor Magical &[7,8, res. \; rule] \\
      Immortal &\lor (\neg Immortal \land Mammal) &[5,6, res. \; rule] \\
      Immortal &\lor Mammal &[10, dist.] \\
      Magical& &[11, 9, res. \; rule] \\
      Horned& &[7, 11, res. \; rule]
    \end{align}

    \item Mythical? : No, Cannot be shown.

    \item Magical? : Yes, Shown above.

    \item Horned? : Yes, Shown above.
  \end{enumerate}

\end{enumerate}

\end{document}
