\documentclass[10pt,draftclsnofoot,onecolumn]{IEEEtran}

\usepackage{setspace}
\usepackage{listings}
\usepackage{cite}
\usepackage{enumerate}
\usepackage{amsthm}
\usepackage[fleqn]{amsmath}
% \usepackage{chngcntr}
% \usepackage{mathtools}

% \numberwithin{equation}

% set syntax highlighting for listings
\lstset{language=C}
\lstset{frame=lrbt,xleftmargin=\fboxsep,xrightmargin=-\fboxsep}

% correct bad hyphenation here
\hyphenation{op-tical net-works semi-conduc-tor}

\begin{document}

\pagenumbering{gobble} % hide page number on first page
\singlespacing % set spacing
\title{Written Assignment \#3}

\author{Ty~Skelton}

% make the title area
\maketitle

\begin{center}
\scshape                      % Small caps
CS331 - Intro to AI \\        % Course
Spring Term\\[\baselineskip]  % Term
Oregon State University\par   % Location
\end{center}

\IEEEpeerreviewmaketitle

\newpage
\pagenumbering{arabic}

\begin{enumerate}
  \item Probability Distributions
  \begin{enumerate}[a)]
    \item \( \textbf{P}(Toothache) \)\\

    \begin{tabular}{|c|c|}
      \hline
      \( Toothache \) & \( \textbf{P}(Toothache) \) \\
      \hline
      False & \( 0.576 + 0.144 + 0.008 + 0.072 = \textbf{0.8} \) \\
      True  & \( 0.064 + 0.016 + 0.012 + 0.108 = \textbf{0.2} \) \\
      \hline
    \end{tabular} \medskip % readability

    \item \( \textbf{P}(Cavity) \) \\

    \begin{tabular}{|c|c|}
      \hline
      \( Toothache \) & \( \textbf{P}(Toothache) \) \\
      \hline
      False & \( 0.576 + 0.144 + 0.064 + 0.016 = \textbf{0.8} \) \\
      True  & \( 0.008 + 0.072 + 0.012 + 0.108 = \textbf{0.2} \) \\
      \hline
    \end{tabular} \medskip % readability

    \item \( \textbf{P}(Toothache | Cavity) \) \\

    \begin{tabular}{|c|c|c|}
      \hline
      \( Toothache \) & \( Cavity \) & \( \textbf{P}(Toothache | Cavity) \) \\
      \hline
      True  & True  & \( (0.012 + 0.108)/0.2 = \textbf{0.6} \) \\
      True  & False & \( (0.064 + 0.016)/0.8 = \textbf{0.1} \) \\
      False & True  & \( (0.008 + 0.072)/0.2 = \textbf{0.4} \) \\
      False & False & \( (0.576 + 0.144)/0.8 = \textbf{0.9} \) \\
      \hline
    \end{tabular} \medskip % readability
  \end{enumerate}

  \item Show Equivalence
  \begin{align*}
    P(X,Y) &= P(X)P(Y): \\
           &= P(X|Y)P(Y) & [Chain rule] \\
           &= (\frac{P(X \land Y)}{P(Y)})P (Y) & [Kolmogorov \; def] \\
           &= (\frac{P(X)P(Y)}{P(Y)})P(Y) & [mult. \; rule \; when \; indep.] \\
           &= P(X)P(Y)& [division] & \qed \\ \\
    P(X|Y) &= P(X): \\
    P(X|Y) &= \frac{P(X \land Y)}{P(Y)} & [Kolmogorov \; def] \\
           &= \frac{P(X)P(Y)}{P(Y)} & [mult. \; rule \; when \; indep.] \\
           &= P(X) & [division] & \qed \\ \\
    P(Y|X) &= P(Y): \\
    P(Y|X) &= \frac{P(Y \land X)}{P(X)} & [Kolmogorov \; def] \\
           &= \frac{P(Y)P(X)}{P(X)} & [mult. \; rule \; when \; indep.] \\
           &= P(Y) & [division] & \qed \\
  \end{align*}

  \newpage % readability

  \item Suppose you are given a coin that lands heads with probability x and tails with probability (1-x).
  \begin{enumerate}[a)]
    \item \textit{Are the outcomes of successive flips of the coin independent of each other given that you know the value of x?} \\ \\
    Regardless of knowing probabilty \( x \) the outcomes of previous coin flips have no effect on the next.

    \item \textit{Are the outcomes of successive flips of the coin independent of each other if you do not know the value of x?} \\ \\
    Like part a, whether or not you know the probability of the coin flip's outcome the successive flips would have no effect on any other.
    Knowing or not knowing doesn't change the probability of an event that is observed.  \\
  \end{enumerate}

  \item After your yearly checkup, the doctor has bad news and good news. \\ \\
  It's good news, because even though the test is highly accurate the probability of having the disease is quite low.
  The actual probability of having the disease given testing positive is:
  \begin{align*}
    P(Have \; | \; Pos) &= \frac{P(Pos \; | \; Have)P(Have)}{P(Pos)} \\
    &= \frac{P(Pos \; | \; Have)P(Have)}{P(Pos \; | \; Have)P(Have) + P(Pos \; | \; Doesn't \; Have)P(Doesn't \; Have)} \\
    &= \frac{(0.99*0.0001)}{(0.99*0.0001)+(0.01*0.9999)} \\
    &= \frac{0.000099}{0.000099+0.009999} \\
    &\approx 0.0098 \\
    &\approx 0.98\% \\
  \end{align*}

  \item Suppose you are witness to a nighttime hit-and-run accident involving a taxi in Athens.

  \begin{enumerate}[a)]
    \item It is not possible to calculate the most likely color of the taxi without more information.
    In order to get this probability we'd need to know the number of blue and green taxis in the city.
    \item Now that we know 9 out of 10 taxis in Athens are green we can calculate the most likely color.
    \begin{align*}
      P(Blue \; | \; Saw \; Blue) &= \frac{P(Saw \; Blue \; | \; Blue)P(Blue)}{P(Saw \; Blue)} \\
      &= \frac{P(Saw \; Blue \; | \; Blue)P(Blue)}{P(Saw \; Blue \; | \; Blue)P(Blue)+P(Saw \; Blue \; | \; Green)P(Green)} \\
      &= \frac{0.75*0.1}{(0.75*0.1)+(0.25*0.9)} \\
      &= 0.25
    \end{align*}

    Since the probability you saw blue and the car was actually blue is only 0.25, it's most likely that you saw blue and the car was actually green!

  \end{enumerate}

\end{enumerate}

\end{document}
